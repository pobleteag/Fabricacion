\section {Definición}

	La pizarra es una roca metamórfica homogenea, formada a partir de otras rocas mediante un proceso de metamorfización. ´´ Transformación sin cambio de estado de la estructura o la composición química o mineral de una roca cuando queda sometida a condiciones de temperatura o presión distintas de las que la originaron o cuando recibe una inyección de fluidos."

	\subsection {Características físicas}
	La principal caracteristica es su dureza, razón por la cual es empleada en la fabricación de mesas de billar.

	La pizzara se suele encontrar en laminas o capas y la inmensa malloría suelen poseer un color negro azulado o grisaceo, pero puede aparecer en tonos  verdes o rojos.

	Otra características es su impermeabilidad, razón por la cual se emplea para pabimentar o techar en lugares de abundantes precipitaciones y temperaturas bajas. 

Su textura es de Granos finos y pizarrosa. Compacta y no sufre meteorización apreciable.

Es muy poco densa, con granos finos formados por metamorfismo de esquisto micáceo, arcilla o, con menor frecuencia, de rocas ígneas

Su brillo es opaco. Satinado.

	\subsection {Características químicas}

La pizarra se compone principalmente de arcilla, ceniza volcanica o con menor frecuencia rocas igneas combinadas con minerales (metamorfización) como, cuarzo y muscovita o illita, a menudo junto con biotita, clorita, hematita y pirita y, con menor frecuencia, apatita, grafito, caolín, magnetita, turmalina o circón, así como de feldespato.

	\subsection {Características mecánicas}
 
Valor del modulo de elasticidad (Farmer, 1968) 1-3,5 $(Kg/cm2) x 10^5$
\par
La siguientes caracteristicas se corresponden con la pizzara de Bernardos (Segovia). Que son las canteras de pizarra más antiguas de España. Se comenzaron a explotar hace 450 años por orden de Felipe II.

 \begin{itemize}
        \item Densidad aparente $2,77 g/cm^3$ 
	\item Resistencia a las heladas $0,01\%$
	\item Resistencia a la flexión $49,41 Mpa$
	\item Resistencia al anclaje $3.979,16 N$
	\item Resistencia al choque $0,05\%$
	\item Resistencia a la compresión $104,69 Mpa$
	\item Absorción de agua $0,23\%$
	\item Resistencia al desgaste por rozamiento $4,48 mm$
	\item Resistencia a los cambios termicos $0,05\%$
	\item Resistencia a los ácidos $<0,02\%$
   \end{itemize}

\clearpage

\section {Estracción/obtención}

	\subsection {Extracción/Síntesis}

	\subsection {Proceso de obtención}
