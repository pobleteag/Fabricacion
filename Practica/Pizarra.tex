\section {Definición}
	La pizarra es una roca metamórfica homogenea, formada a partir de otras rocas mediante un proceso de metamorfización.

	\subsection {Características físicas}
	La principal caracteristica es su dureza, razón por la cual es empleada en la fabricación de mesas de billar.

	La pizzara se suele encontrar en laminas o capas y la inmensa malloría suelen poseer un color negro azulado o grisaceo, pero puede aparecer en tonos  verdes o rojos.

	Otra características es su impermeabilidad, razón por la cual se emplea para pabimentar o techar en lugares de abundantes precipitaciones y temperaturas bajas. 
	\subsection {Características químicas}

	\subsection {Características mecánicas}

\section {Estracción/obtención}

	\subsection {Extracción/Síntesis}

	\subsection {Proceso de obtención}
