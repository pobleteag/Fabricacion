\section {Definición}

	La pizarra es una roca metamórfica homogenea, formada a partir de otras rocas mediante un proceso de metamorfización. ``Transformación sin cambio de estado de la estructura o la composición química o mineral de una roca cuando queda sometida a condiciones de temperatura o presión distintas de las que la originaron o cuando recibe una inyección de fluidos."

	\subsection {Características físicas}
	La principal característica es su dureza, razón por la cual es empleada en la fabricación de mesas de billar.

	La pizarra se suele encontrar en laminas o capas y con  un color negro azulado o grisáceo en la mayoría de los casos, pero puede aparecer en tonos  verdes o rojos.

	Otra características es su impermeabilidad, razón por la cual se emplea para pavimentar o techar en lugares de abundantes precipitaciones y temperaturas bajas. 

Su textura es de Granos finos y pizarrosa. Compacta y no sufre meteorización apreciable.

Es muy poco densa, con granos finos formados por metamorfismo de esquisto micáceo, arcilla o, con menor frecuencia, de rocas ígneas

Su brillo es opaco. Satinado.

	\subsection {Características químicas}

La pizarra se compone principalmente de arcilla, ceniza volcánica o con menor frecuencia rocas ígneas combinadas con minerales (metamorfización) como, cuarzo y muscovita o illita, a menudo junto con biotita, clorita, hematita y pirita y, con menor frecuencia, apatita, grafito, caolín, magnetita, turmalina o circón, así como de feldespato.

	\subsection {Características mecánicas}
 
Valor del modulo de elasticidad (Farmer, 1968) 1-3,5 $(Kg/cm2) x 10^5$
\par
La siguientes características se corresponden con la pizarra de Bernardos (Segovia). Que son las canteras de pizarra más antiguas de España. Se comenzaron a explotar hace 450 años por orden de Felipe II.

 \begin{itemize}
        \item Densidad aparente $2,77 g/cm^3$ 
	\item Resistencia a las heladas $0,01\%$
	\item Resistencia a la flexión $49,41 Mpa$
	\item Resistencia al anclaje $3.979,16 N$
	\item Resistencia al choque $0,05\%$
	\item Resistencia a la compresión $104,69 Mpa$
	\item Absorción de agua $0,23\%$
	\item Resistencia al desgaste por rozamiento $4,48 mm$
	\item Resistencia a los cambios termicos $0,05\%$
	\item Resistencia a los ácidos $<0,02\%$
   \end{itemize}

\clearpage

\section {Estracción & obtención}
Debido a su composición la pizarra suele estar en zonas volcánicas.

Gales es una de las zonas con mayor producción de pizarra del mundo. En Portugal, Italia o Alemania también hay importantes explotaciones de pizarra. En España los yacimientos importantes están en El Bierzo,
Cabrera y Valdeorras (León y Orense), Bernardos (Segovia) y Villar del Rey (Badajoz).

En América encontramos lugares de extracción en Brasil, siendo el segundo productor del mundo. También en Norteamérica hay zonas de producción importantes en Terranova, y Nueva York.

	\subsection {Extracción}

La maquinaria pesada juega un papel primordial en este proceso.

La extracción de pizarra se realiza con ayuda  de sierras de disco
diamantado. Estas realizan cortes en lineas perpendiculares, dependiendo de las características
de la pizarra y del lugar.

La separación de las placas serradas se realiza a través de cuñas
mecánicas, adaptadas a las palas cargadoras. Para retirar las
placas, se utiliza una pala cargadora con enganche rápido,
substituyendo a la caja (pala cargadora con función de apiladora).
Las placas son colocadas directamente sobre los camiones, y transportadas hasta la fábrica,
donde, la piedra es seleccionada y abierta manualmente, sin utilizar maquinaria o cualquier
fuerza mecánica.

La piedra se abre fácilmente, empleado tan solo una cuña y un cincel para ello. Esta forma de
abrirla mantiene las características naturales de la piedra favoreciendo la calidad y durabilidad
de estas.

